\documentclass{DISERTASI-ITS}
% ===========================================================================================
% Template ini masih dalam tahap pengembangan, jika ada perbaikan bisa dimasukkan ke github: https://github.com/arimogi/Template-Disertasi-PD-Ilmu-Komputer-ITS-Surabaya-Indonesia
% ===========================================================================================

% tambahkan package yang dibutuhkan di sini
% \numberwithin{equation}{chapter}
% untuk teorema, jika diperlukan
% \theoremstyle{plain}
% \newtheorem{thm}{Teorema}[section]
% \newtheorem{prop}[thm]{Proposisi}
% \newtheorem{lemma}[thm]{Lemma}
% \newtheorem*{cor}{Akibat}
% \theoremstyle{remark}
% \newtheorem{rmk}{Remark}
% \renewcommand{\qedsymbol}{$ \blacksquare $}
% \theoremstyle{definition}
% \newtheorem{defi}{Definisi}[section]
	
\usepackage{datetime}
	\newdateformat{monthdayyeardate}{%
	\THEDAY~\monthname[\THEMONTH]~\THEYEAR}

\usepackage{caption} 
\captionsetup{margin=0pt,singlelinecheck=off,format=plain,indention=.15cm,justification=raggedright}

\begin{document}
    % hanya untuk proposal tambahkan keterangan ini.
    \Keterangan{proposal} % kalau kosong berarti Keterangan pada lembar Judul diberi keterangan Disertasi, selain itu berarti Proposal Disertasi, 

    \Judul{Judul Bahasa Indonesia} 	            % Tuliskan Judul Yang sesuai (bhs.Indonesia)
    \JudulEng{Judul Bahasa Inggris} 			% Tuliskan Judul Yang seuai (bhs.Inggris)
    \Nama{Nama Mahasiswa} 						% Tuliskan Nama penulis proposal tesis
    \NRP{XXXXXXXX} 							    % Tuliskan Nrp penulis proposal tesis
    \Jurusan{Informatika}						% Tuliskan Departement Penulis tesis
    \Department{Informatics}					% Tuliskan Departement Penulis tesis (bhs. Inggris)
    \BidangStudi{Analisis dan Aljabar} 			% Tuliskan bidang sesui minat penulis
    \Tahun{2022} 								% Tuliskan tahun pembuatan proposal tesis
    \Fakultas{ELECTICS} 		% Tuliskan Fakultas pembuatan proposal tesis
    \Faculty{ELECTICS} 		% Tuliskan Fakultas pembuatan proposal tesis (bhs. Inggris)
	
    \KepalaDep{Kepala Departemen, S.Si., M.Sc., Ph.D.}
    \NipKepalaDep{196001311990021001}

    % ===============================
    % Pembimbing
    % ===============================
                                                % contoh: (Dr. Pembimbing, S.Kom., M.T.)
    \Pembimbing{Prof. Dr. Pembimbing, M.S.} 		% Tuliskan nama pembimbing 1
    {Dr. Pembimbing Dua, M.Si.} 					% Tuliskan nama pembimbing 2, kosongkan bila tidak ada
    {}          							    % Tuliskan nama pembimbing 3, kosongkan bila tidak ada

                                                % contoh: (198305172008121003)
    \NIPPembimbing{196001011980031001} 			% Isi dengan NIP pembimbing 1
    {198001012000011001} 						% Isi dengan NIP pembimbing 2 bila ada
    {}    										% Isi dengan NIP pembimbing 3 bila ada

    % ===============================
    % Penguji
    % ===============================
    \Penguji{Dr. Penguji Satu, M.Si.}   	% Isi nama penguji  1  yang sesuai
    {Dr. Dra. Penguji Dua, M.Si.}           	% Isi nama penguji  2  yang sesuai
    {Dr. Penguji Tiga S, S.Si., M.T.}          % Isi nama penguji  3  yang bila ada 
    	
    \NIPPenguji{1960xxxxxxxxxxxxxx}  	% Isi Nip penguji 1 yang sesuai
    {1960xxxxxxxxxxxxxx}             	% Isi Nip penguji 2 yang sesuai
    {1960xxxxxxxxxxxxxx}             	% Isi Nip penguji 3 yang sesuai bila ada bila tidak kosongkan

    %		
    \TanggalDisetujui{6 Desember 2020} %  \monthdayyeardate\today   %:tanggal pelaksanaan seminar proposal (21 September 2020)  5 Desember 2020
    \TanggalSeminar{4 Januari 2023}
    \HariSeminar{Rabu}
    \TempatSeminar{Ruang Sidang II Departemen Matematika, Gedung U Lantai 2}
    \TanggalUjian{11 Februari 2023}
    \PeriodeWisuda{Dua}
    
    \BagianAwal
    
    \LembarJudul  % menampilkan halaman judul
    \TitlePage
    %\LembarPersetujuan % Menampilkan Lembar Persetujuan Seminar oleh Promotor
    \LembarPengesahan % Menampilkan Lembar Pengesahan Seminar Proposal oleh promotor dan penguji
    %\LembarPengesahanDisertasi % Menampilkan Lembar Pengesahan Disertasi

    %==================
    %  ABSTRAK
    %==================    
    \begin{Abstrak} % Abstrak bhs. Indonesia
        
Abstrak Bahasa Indonesia

\medskip

\katakunci{kata kunci 1, kata kunci 2} % isikan foder dan file abstrakind.tex
    \end{Abstrak}    	
    \begin{Abstract} % Abstrak bhs. Inggris
        Abstrak Bahasa Inggris

\medskip

\keywords{Keyword 1, keyword 2} %  isikan folder dan file abstrakenglish.tex
    \end{Abstract}

    %===============================
    %  Daftar Isi, Gambar, Tabel
    %===============================
    \DaftarIsi % menampilkan secara otomatis semua Daftar Isi tesis
    \DaftarGambar
    \DaftarTabel

    %===============================
    %  Isi Disertasi
    %===============================    
    \BagianInti
    %========>  Proposal Disertasi
    \chapter{PENDAHULUAN}
Pada bab ini, dijelaskan beberapa hal yang menjadi dasar dalam penyusunan proposal disertasi. Dari ulasan tersebut kemudian dirumuskan permasalahan yang akan dibahas serta tujuan dan kontribusinya \cite{Moody1998}.

\section{Latar Belakang}
\section{Perumusan Masalah}
\section{Tujuan dan Manfaat Penelitian}

 % isikan nama folder dan file BabI.tex sebagai Bab pendahuluan    
    \chapter{KAJIAN PUSTAKA DAN DASAR TEORI}
\section{Kajian Pustaka}
\section{Dasar Teori} % isikan nama folder dan file BabII.tex  
    % !TeX spellcheck = id
\chapter{METODE PENELITIAN}

\section{Tahapan Penelitian}

\section{Rencana dan Jadwal Kegiatan Penelitian} % isikan nama folder dan nama file Bab3.tex    
    
    %========>  Disertasi Lengkap, kosongkan jika masih proposal
    \chapter{HASIL}


    \chapter{PENUTUP}



    %======================================
    %  Penutup, Bagian akhir disertasi
    %======================================
    \DaftarPustaka{ReferensiDisertasi}	
    	
\end{document}
